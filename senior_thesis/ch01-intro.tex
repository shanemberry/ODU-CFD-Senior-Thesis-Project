\documentclass[mythesis.tex]{subfiles}
\begin{document}
% tells latex when compiling a subfile what chapter number this document is
% set the chapter to minus one as \chapter will increment the counter
\setcounter{chapter}{0}
\chapter{Introduction}

With the advent of CEBAF, it will become routine to probe nuclear
systems with electron scattering where the energy and momentum
transfers will be well in excess of the nucleon
mass. Under such circumstances, the usual
nonrelativistic description of the nucleus is no longer reliable. It is,
therefore, necessary to develop relativistically covariant models of the
nuclear system. This is a difficult task and most work has been concentrated
on the two-nucleon system, although extensions to three-body systems are
being studied.

There are basically three approaches to the construction of covariant models
of the deuteron: relativistic Hamiltonian dynamics
\cite{KeisterandPolyzou,Chung}, Bethe-Salpeter
\cite{BetheSalpeter,Tjon,Zuilhof,Umnikov} and related quasipotential equations
\cite{BSLT,Thompson,Todorov,Erkelenz,Kadychevsky,GrossA,GrossB,FVOHA,FVOHB,BuckandG,ACG,Hummel,WallaceandM,DevineandW,Zhu},
and light-cone field theory \cite{Fuda,Karmanov}. The first of these
concentrates on the application of Poincare invariance to Hamiltonian
theories with potential-like interactions. The Bethe-Salpeter and light-cone
field theories are derived from field theory, although actual applications
to two-nucleon systems require the introduction of phenomenological
elements. The Bethe-Salpeter equation and the related quasipotential
equations are based on Feynman perturbation theory and, as such, maintain
manifest Lorentz invariance. The light-cone field theory approach is based
on the evolution of field theories quantized along the light-cone. Both the
Hamiltonian dynamics and the light-cone field-theory focus on the ``time''
evolution of interacting systems and are organized such that the
calculations are covariant but not necessarily manifestly covariant.

This paper focuses on the application on Bethe-Salpeter and quasipotential
approaches to the modeling of the deuteron. The actual calculation of these
models can become quite complex due to the spin degrees of freedom of
mesons and nucleons. For this reason, it is instructive to first consider
these equations in the context of a simple model containing only scalar
particles. The basic structure of these equations is easily described
and the calculation of bound states and scattering amplitudes is
considerably less formidable. This allows the solutions to be examined
for a variety of quasipotential equations. Some interesting qualitative
features of these calculations can be identified that will carry over into
more realistic models. In particular, it is shown that the scattering phase
shifts seem to be relatively insensitive to the choice of quasipotential
equation provided that the model parameters are appropriately adjusted to
obtain a bound state at a fixed mass.
The spectator or Gross equation is chosen to
demonstrate the problems associated with the calculation of electromagnetic
current matrix elements in quasipotential models. This model is used to
calculate the elastic electron scattering form factor for the scalar
``deuteron'' and the effects of variation of nucleon cutoffs and
approximations to the single-nucleon current operator are studied.

The basic concepts introduced in the context of the scalar model are
extended to cover  more realistic models of two spin-$\frac{1}{2}$ nucleons
interacting through the exchange of mesons with a variety of masses,
spins and isospins. Some of the additional complications arising from
these additional internal quantum numbers are introduced. Two interaction
models are discussed and applied to the calculation of the structure
functions for elastic electron scattering from the deuteron. The
calculations are in good agreement with the data available for these
structure functions.

\end{document}
