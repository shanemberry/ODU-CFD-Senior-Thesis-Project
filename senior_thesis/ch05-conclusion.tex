\documentclass[mythesis.tex]{subfiles}
\begin{document}
% tells latex when compiling a subfile what chapter number this document is
% set the chapter to minus one as \chapter will increment the counter
\setcounter{chapter}{4}
\chapter{Conclusions}

This paper provides an introduction to the Bethe-Salpeter and related
quasipotential equations. A simple scalar model is introduced to show the
features of the solutions of these equations. This model shows that,
provided the coupling is adjusted to reproduce the bound-state mass, the
phase shifts display little variation among a representative collection
of quasipotential equations. This feature needs to be studied in more
detail for more complicated and realistic models. Care must be taken in
constructing the electromagnetic current matrix elements for these covariant
equation in order that gauge invariance not be violated.

Preliminary calculations are presented for the elastic structure functions
of the deuteron using realistic meson-nucleon models. These models provide
a reasonable description of the structure functions. In particular the
calculation of $B(Q^2)$ using model IIB is the best representation of this
structure function yet obtained in the context of such covariant models.

Considerable work remains to be done in extending these calculations to
other reactions and in the study of the relative characteristics of the
various relativistic approaches.

\end{document}
